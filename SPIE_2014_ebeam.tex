%  The following commands have been added in the SPIE class 
%  file (spie.cls) and will not be understood in other classes:
%  \supit{}, \authorinfo{}, \skiplinehalf, \keywords{}
%  The bibliography style file is called spiebib.bst, 
%  which replaces the standard style unstr.bst.  
\documentclass[]{spie}  %>>> use for US letter paper
%%\documentclass[a4paper]{spie}  %>>> use this instead for A4 paper
%%\documentclass[nocompress]{spie}  %>>> to avoid compression of citations
%% \addtolength{\voffset}{9mm}   %>>> moves text field down
%% \renewcommand{\baselinestretch}{1.65}   %>>> 1.65 for double spacing, 1.25 for 1.5 spacing 
%  The following command loads a graphics package to include images 
%  in the document. It may be necessary to specify a DVI driver option,
%  e.g., [dvips], but that may be inappropriate for some LaTeX 
%  installations. 
\usepackage[]{graphicx}

\title{High Performance Si Immersion Gratings Patterned with Electron Beam Lithography} 

%>>>> The author is responsible for formatting the 
%  author list and their institutions.  Use  \skiplinehalf 
%  to separate author list from addresses and between each address.
%  The correspondence between each author and his/her address
%  can be indicated with a superscript in italics, 
%  which is easily obtained with \supit{}.

\author{Michael Gully-Santiago\supit{a}, Daniel T. Jaffe\supit{a}, Dan Wilson\supit{b}, Richard Muller\supit{b}
\skiplinehalf
\supit{a}University of Texas at Austin Department of Astronomy, 2515 Speedway St, Austin, TX, USA; \\
\supit{b}NASA JPL Microdevices Lab, 4800 Oak Grove Dr, Pasadena, CA, USA
}

%>>>> Further information about the authors, other than their 
%  institution and addresses, should be included as a footnote, 
%  which is facilitated by the \authorinfo{} command.

\authorinfo{Further author information: (Send correspondence to M.G.S)\\M.G.S: E-mail: gully@astro.as.utexas.edu}
%%>>>> when using amstex, you need to use @@ instead of @
 

%%%%%%%%%%%%%%%%%%%%%%%%%%%%%%%%%%%%%%%%%%%%%%%%%%%%%%%%%%%%% 
%>>>> uncomment following for page numbers
% \pagestyle{plain}    
%>>>> uncomment following to start page numbering at 301 
%\setcounter{page}{301} 
 

\begin{document} 
  \maketitle 

%%%%%%%%%%%%%%%%%%%%%%%%%%%%%%%%%%%%%%%%%%%%%%%%%%%%%%%%%%%%% 
\begin{abstract}
Infrared spectrographs employing silicon immersion gratings can be significantly more compact than spectrographs using front-surface gratings.  The Si gratings also offer the possibility of continuous wavelength coverage at high spectral resolution.  The grooves in Si gratings are made with semiconductor lithography techniques, to date almost entirely using contact mask photolithography.  Planned near-infrared astronomical spectrographs can require either finer groove pitches or higher positional accuracy than can be achieved with standard UV contact mask photolithography.  A collaboration between the University of Texas at Austin Silicon Diffractive Optics Group and the NASA JPL Microdevices Lab has experimented with direct writing silicon immersion grating grooves with electron beam lithography.  The device production process involves depositing positive e-beam resist on 1 to 30 mm thick, 100 mm diameter monolithic crystalline silicon substrates.  The JEOL 9300FS e-beam writer uses a 50 nm step size with a typical spot size of 300 nm at about 60 nA of current and 100 keV power.  Our typical groove frequency for echelle grating prototypes is in the vicinity of 40 $-$ 12.5 grooves/mm (25-80 $\mu$m groove spacing) but groove frequencies as high as 1000 grooves/mm are possible.  Experimental pattern sizes are up to 80 $\times$ 30 mm$^2$, but could exceed 200 $\times$ 200 mm$^2$.

There are three key challenges to produce high-performance e-beam written silicon immersion gratings.  (1) E-beam field and subfield stitching boundaries cause periodic cross-hatch structures along the grating grooves.   The structures manifest themselves as spectral and spatial dimension ghosts in the diffraction limited point spread function (PSF) of the diffraction grating.  In this paper, we show that the effects of e-beam field boundaries must be mitigated.  We have significantly reduced ghost power with only minor increases in write time by using four or more reticles of less than 500 $\mu$m. (2) The finite e-beam stage drift and run-out error cause large-scale structure in the wavefront error.  We deal with this problem by applying a mark detection loop to check for and correct out minuscule stage drifts.  We measure the level and direction of stage drift and show that mark detection reduces peak-to-valley wavefront error by a factor of 5. (3) The serial write process for typical gratings yields write times of about 24 hours- this makes prototyping costly.  We discuss work with negative e-beam resist to reduce the fill factor of exposure, and therefore limit the exposure time.
We also discuss the tradeoffs of long write-time serial write processes like e-beam with UV photomask lithography.  We show the results of experiments on small pattern size prototypes on Silicon wafers.  Current prototypes now exceed 30 dB of suppression on spectral and spatial dimension ghosts compared to monochromatic spectral purity measurements of the backside of Si echelle gratings in reflection at 632 nm.  We perform interferometry at 632 nm in reflection with a 25 mm circular beam, projected to 25 $\times$ 80 mm on the grating surface for a blaze angle of 71.6$^\circ$.  The measured wavefront error (tilt and focus removed) is 0.17 waves peak to valley.
\end{abstract}

%>>>> Include a list of keywords after the abstract 

\keywords{Manuscript format, template, SPIE Proceedings, LaTeX}

%%%%%%%%%%%%%%%%%%%%%%%%%%%%%%%%%%%%%%%%%%%%%%%%%%%%%%%%%%%%%
\section{INTRODUCTION}
\label{sec:intro}  % \label{} allows reference to this section

At the last SPIE meeting we described the detailed performance of the solitary immersion grating (part number CA-1 \cite{2012SPIE.8450E..2SG}) for the high resolution infrared spectrograph IGRINS \cite{2010SPIE.7735E..54Y}.  IGRINS saw first light at McDonald Observatory on March 15, 2014.  Papers describing its performance appear in the current volume.  The technical readiness of the immersion grating now rests on firm footing, and our group has now moved on to pushing the performance and design limitations of silicon optics.  The key limitation is the immersion grating phase coherence.  Phase coherence for a diffraction grating is the extent to which repeated grating facets are positioned to the an integer value of $\sigma$, the groove constant.  Specifically, the position, $x$ of the $n^{th}$ facet in a sequence of $N$ total facets is distributed as $x = n\sigma + \epsilon$, where $\epsilon_n$ is the position error for facet $n$.  Discussion of the phase performance can be broadly separated into phase errors on large scale (low order aberrations), and small scale (spectral ghosts and grass).  In Section \ref{sec}

\section{IGRINS first light affirms technical readiness of Si immersion gratings}
\section{Motivations for direct writing Si immersion gratings}
\subsection{Higher precision than contact lithography}
\subsection{Finer groove pitch limitation than contact lithography}
\section{e-beam patterning strategies}
\subsection{e-beam patterning geometry hierarchy}
\subsubsection{e-beam step size and spot size}
\subsubsection{Subfield deflector}
\subsubsection{Field deflector}
\subsubsection{Interplay of hierarchy at boundaries}
\subsection{Predicting the ghost level associated with field positioning errors}
\subsection{Employ multiple fields sizes rather than one field size}
\subsection{Is there a need for multiple cross-dispersion field sizes?}
\section{Strategies for mitigating facet position errors attributable to large scale stage drift}
\subsection{Characterization of e-beam stage drift}
\subsection{Improvement in wavefront performance from drift correction}
\section{Limitations of direct writing Si immersion gratings}
\subsection{Serial write process yields long write times}
\subsection{Experiments with negative resist}
\subsection{Large per-part investment cost yields heightened process risk}
\section{Conclusions}




\bibliography{SPIE_2014_ebeam}   %>>>> bibliography data in report.bib
\bibliographystyle{spiebib}   %>>>> makes bibtex use spiebib.bst

\end{document} 
